%%%%%%%%%%%%%%%%%%%%%%%%%%%%%%%%%%%%%%%%%
% Beamer Presentation
% LaTeX Template
% Version 1.0 (10/11/12)
%
% This template has been downloaded from:
% http://www.LaTeXTemplates.com
%
% License:
% CC BY-NC-SA 3.0 (http://creativecommons.org/licenses/by-nc-sa/3.0/)
%
% Modified by Jeremie Gillet in March 2017 to make an OIST template
%
%%%%%%%%%%%%%%%%%%%%%%%%%%%%%%%%%%%%%%%%%

% !!!!!
%    I HEREBY SWEAR NOT TO USE THIS TEMPLATE FOR ANY MEANS WHICH WOULD 
%    IN ANY WAY LEAD TO A MONETARY GAIN
% !!!!!

% Author (aside from template, obviously): Vojtech Kalis (xkalis03), presentation created 
%                                                                    for a project for presentation of BT, VUTBR FIT

%----------------------------------------------------------------------------------------
%	PACKAGES AND THEMES
%----------------------------------------------------------------------------------------

\documentclass[10pt]{beamer}

\usepackage{graphicx} % Allows to include images
\usepackage{booktabs} % Allows the use of \toprule, \midrule and \bottomrule in tables
\usepackage{textcomp}

\usepackage{animate} %For gifs (split into pngs)
%\usepackage{multimedia} %For videos

%magick was used to convert gifs to pngs - https://imagemagick.org/index.php
%$ magick convert -coalesce Game_of_Life.gif Game_of_Life.png
%$ magick convert -coalesce SmoothLife.gif SmoothLife.png

\mode<presentation> {

\usetheme{default}

\usecolortheme[named=white]{structure} % White titles and such
\setbeamercolor{normal text}{fg=white} % White text
\setbeamercolor{background canvas}{bg=black} % Black background
\setbeamertemplate{itemize item}{\color{white}$\bullet$} % Comment this line for default bullet points (triangles)
\usepackage{helvet}
\renewcommand{\familydefault}{\sfdefault}

\setbeamertemplate{navigation symbols}{} % No navigation symbols
\setbeamertemplate{footline}
 {\begin{minipage}{125mm} \vspace{-4 mm} \hfill \textbf{\normalsize{\insertframenumber\,/\,\inserttotalframenumber}} \end{minipage}}
 }

%----------------------------------------------------------------------------------------
%	TITLE PAGE SETUP
%----------------------------------------------------------------------------------------

\vspace*{-1cm}
\title[Short title]{BT - Simulace Biologických Procesů pomocí Asynchronních Celulárních Automatů a Strojového Učení\\\vspace{0.5cm}}
\subtitle{}
\author{Vojtěch Kališ, xkalis03\\\vspace{0.2cm}Vedoucí práce: Ing. Karel Fritz}
\date{}

%----------------------------------------------------------------------------------------
%	PRESENTATION SLIDES
%----------------------------------------------------------------------------------------

\begin{document}


%------------------------------------------------
%	TITLE PAGE
\setbeamertemplate{background}{\includegraphics[width=\paperwidth, trim = 0 0 0 -17]{img/title.png}} % Adding the background logo for the title page

\begin{frame}[plain]
\vspace*{1.55cm}
\titlepage
\end{frame}

\setbeamertemplate{background}{\includegraphics[width=\paperwidth]{img/background.png}} % Adding the background logo for the rest of the slides

%------------------------------------------------
%	GOALS
\begin{frame}
	\frametitle{Cíle práce}
	\begin{itemize}[<+->]
		\pause
		\item Celulární Automaty a Strojové Učení
			\begin{itemize}[<2->]
				\item individuální studie
				\item sloučení konceptů + benefity
			\end{itemize}
		\pause
		\vspace{0.2cm}
		\item Emergence, Emergentní chování
			\begin{itemize}[<4->]
				\item přehlížené téma
				\item obtížnost studie
				\item benefity
			\end{itemize}
		\end{itemize}
\end{frame}

%------------------------------------------------
%	PROPOSED SOLUTION
\begin{frame}
	\frametitle{Návrh řešení}
	Uvedení do tématu a prezentace výsledků
	\vspace*{0.5cm}
	\begin{itemize}[<+->]
		\pause
		\item Conway's Game of Life
			\begin{itemize}[<2->]
				\item základní báze znalostí
			\end{itemize}
		\pause
		\vspace{0.2cm}
		\item SmoothLife
			\begin{itemize}[<4->]
				\item kontinuální stavový prostor
				\item rozšířené okolí buňky
			\end{itemize}
		\pause
		\vspace{0.2cm}
		\item Lenia
			\begin{itemize}[<6->]
				\item asynchronismus + kontinuální čas
				\item zakomponování strojového učení - emergence
			\end{itemize}
	\end{itemize}
\end{frame}

%------------------------------------------------
%	CURRENT PROGRESS - GAME OF LIFE
\begin{frame}
	\frametitle{Informace o stavu řešení}
	\centering
	\parbox{\textwidth}{
		\centering
		Conway's Game of Life
		\vspace{0.2cm}
	}
	\parbox{0.6\textwidth}{
		\animategraphics[width=0.6\textwidth, autoplay, loop]{8}{gif/Game_of_Life-}{0}{456}
	}
\end{frame}

%------------------------------------------------
%	CURRENT PROGRESS - SMOOTHLIFE
\begin{frame}
	\frametitle{Informace o stavu řešení}
	\centering
	\parbox{\textwidth}{
		\centering
		SmoothLife
		\vspace{0.2cm}
	}
	\parbox{0.6\textwidth}{
		\animategraphics[width=0.6\textwidth, autoplay, loop]{8}{gif/SmoothLife-}{0}{199}
	}
\end{frame}

%------------------------------------------------
%	LITERATURE
\begin{frame}
	\frametitle{Přečtená literatura}
	\begin{itemize}
		\item "Ordered asynchronous processes in multi-agent systems", 2005 - D. Cornforth, D. G. Green, D. Newth
		\item "Growing Neural Cellular Automata", 2020 - A. Mordvintsev, E. Randazzo, E. Niklasson, M. Levin
		\item "Cellular Automata Models in Biology", 1990 - D. G. Green
		\item "Studying Growth with Neural Cellular Automata", 2022 - S. Greydanus
		\item "Asynchronous cellular automata", 2017 - Nazim A. Fatès
		\item "Lenia - Artificial Life from Algorithms" - Birdbrain (https://www.youtube.com/watch?v=6kiBYjvyojQ)
	\end{itemize}
\end{frame}

%------------------------------------------------

\end{document} 